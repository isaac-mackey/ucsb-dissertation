\begin{abstract}
\addcontentsline{toc}{chapter}{Abstract}
%todo: max 350 words
%current: 378 words

The rise of automated systems and sensor networks
in virtually all areas of industry and social life
means many technologies produce streams of events rich with information.
These technologies demand algorithms
for evaluating queries on streams,
coordinating systems that communicate with events,
and monitoring streams with respect to specified constraints.
In monitoring,
constraints that define correct behavior,
e.g., business goals, legal requirements, resource limitations, or safety and security concerns
are specified in a formal language;
then,
an event stream is analyzed at runtime
to determine if the constraints are satisfied or violated.
To make monitoring effective,
it is important to detect constraint violations at the earliest possible time,
which we call the early violation detection problem. 

We study early violation detection for a class of constraints called rules
that restrict time gaps between events and compare events' data contents.
We show that (1) the general problem of early violation detection
for an arbitrary set of rules is unsolvable
and 
(2) early violation detection is possible
for various subclasses of rules.
For (1),
we show early violation detection
is closely related to the problem of finite satisfiability
(whether or not a given set of rules can be satisfied by a finite event stream)
and prove that finite satisfiability for a set of rules is undecidable
with a reduction from the empty-tape Turing machine halting problem,
which implies that early violation detection is unsolvable in general.
For (2), we study restricted classes of rules.
A recent proof of Kamp's Theorem provides a translation algorithm
for ``dataless'' rules through translation to linear temporal logic,
yielding formulas hyper-exponential in the size of the input rule.
We present translation algorithms for two subclasses of dataless rules,
improving the output size
from hyper-exponential to single- and double-exponential, respectively.
For rules with data, we first present a technique
that calculates deadlines from time gaps between events,
then use deadlines for early violation detection for individual rules.
We extend these algorithms to monitor an acyclic set of rules
by applying a chase process and satisfiability testing.
We also report the performance of an implementation of these algorithms.
Finally, we consider acyclic sets of rules with aggregation functions over time windows,
combining the chase and satisfiability techniques
with an encoding of aggregation functions in Presburger arithmetic.


%\abstractsignature
\end{abstract}


% This approach is more efficient,
% but requires the ability to
% incrementally update the partial results of previous evaluations,
% which is a non-trivial problem
% given the complexity of business rules.
% One approach to address this problem
% is to use languages with restricted expressiveness,
% such as linear temporal logic (LTL),
% which can be evaluated incrementally,
% but are not expressive enough to capture
% constraints on data in events.
% Thus, effective and efficient techniques
% for expressive business rules
% are an open problem for research.

% A variety of techniques for
% incremental violation detection
% already exist.
% For temporal constraints on propositional events,
% violation detection has often been achieved
% by translating constraints
% into finite state machines.
% We use this approach for rules
% that do not refer to data in events
% and compare our results to an existing translation algorithm.
% For constraints with data
% and for constraints on aggregation functions,
% violation detection can be achieved by creating
% data structures that can be incrementally updated
% and queried to report violations;
% we use this approach for rules with data,
% using a relational database as the data structure
% and reducing the violation detection problem to
% a satisfiability problem for Presburger formulas.

