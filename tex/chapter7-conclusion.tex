\chapter{Conclusion}
\label{chapter:conclusion}

Driven by the demand for 
efficient and effective monitoring of event streams,
we investigate
early violation detection for business rules.
Enforcing these constraints
at runtime,
rather than at design-time,
allows enables the enforcement
of business rules in event-based systems,
while avoiding intractable design-time analysis.
In this dissertation,
we show that
early violation detection for sets of rules is unsolvable in general,
indicating that more research
is needed to identify tractable cases.
We also contribute
algorithms to solve restricted cases
of the early violation detection problem,
relying on techniques from
business process management, automated reasoning,
and database systems.
We study the translation of dataless rules to LTL formulas,
improving the output size of the best known translation
for two subclasses of rules.
We consider the problem for individual rules and
acyclic sets of rules with data,
applying a chase process and satisfiability testing,
then add aggregation functions on time windows to the rule language.
These techniques are novel in their application to early violation detection, especially in the context of quantitative time constraints
and incremental monitoring of event streams. 

More work is needed to understand
violation detection problems.
This includes
the study of richer classes of temporal constraints,
beyond the gap constraints considered in this dissertation.
Additionally,
rules lack some features common in natural language and compliance regulations
that may be useful in modeling real-world constraints,
e.g.,
negation for modeling the absence of events
and
disjunction for modeling choice or multiple possibilities.
More work is needed
to determine how these features affect the complexity of the early violation detection problem.

Also,
our techniques consider only perfect data 
and
whether or not a violation is certain.
Realistic event streams
may also contain some noise and uncertainty,
so reporting violations probabilistically
and 
relaxing assumptions about the event stream's quality,
e.g., allowing out-of-order events,
deserve further study.
Finally,
the undecidability proof suggests that
the border between solvable and unsolvable problems
can be made sharper
with respect to the number and type of event attributes.
In summary,
our insights improve
the understanding of the early violation detection problem
and may guide the design of rule monitors for event-based systems,
but more work is needed
to understand the full range of effective early violation detection.