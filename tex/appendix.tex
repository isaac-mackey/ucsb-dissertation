% \chapter{Appendix}{\label{appendix:a}}
% \begin{section}{Datalog Program for Aggregation Events}

% \subsection{Moving window with {\sc count} function}

% Let \texttt{Src} be the source event and $L+1$ the window size for a moving rule
% with the {\sc count} function,
% i.e., the rule has the form of Fig.\:\ref{fig:moving-count-rule}.

% \begin{figure}[h!]
% \begin{tabular}{ll}
% \texttt{resultMovingCount}$(\mbox{\sc count}(a),s)@(s+L) \leftarrow$    & $\textsc{Over}\ {\sc Moving}(s,s+L)$\\
%                                                                             & $\textsc{From}\ \mbox{\tt Src}(a)@x$
% \end{tabular}\\
% \caption{Rule for moving window with {\sc count} function}
% \label{fig:moving-count-rule}
% \end{figure}

% To compute \texttt{resultMovingCount} incrementally,
% we use an internal event \texttt{incMovingCount}
% with attributes {\em accumulator},
% {\em start},
% {\em end},
% and {\em stop}.
% The {\em start} attribute is the window's first timestamp.
% The {\em accumulator} attribute
% holds the count seen from {\em start} to the timestamp {\em stop},
% where {\em stop} is the latest timestamp for the window seen thus far
% and used as each \texttt{incMovingCount} event's timestamp.
% The {\em end} attribute is the window's last timestamp
% and indicates when the result should be reported.\\

% We use a Datalog program (Fig.\:\ref{fig:moving-count-infty})
% to generate \texttt{incMovingCount} events as follows:
% each timestamp is the start of a moving window
% so for each source event at timestamp $T$,
% an \texttt{incMovingCount} event is initialized
% using a source event's timestamp as the window start,
% a $1$ in the accumulator if a ``genuine'' source event is present,
% i.e., if the source event value is not $-\infty$,
% otherwise a $0$,
% and $T{+}L$ as the window end.
% Data in \texttt{incMovingCount} at a timestamp $T$ is propagated
% to the next timestamp $T{+}1$ using the source event at timestamp $T{+}1$.
% To do this, the accumulator is incremented if the source event is real,
% and unchanged otherwise.
% The propagation only happens if the next timestamp is no greater than the window end.
% When the source timestamp is equal to the window end,
% the value in the accumulator is reported with \texttt{resultMovingCount}.\\

% \begin{figure}[h!]
% \begin{mdframed}[leftmargin=0pt,rightmargin=0mm]
% \begin{small}
% \begin{tabular}{ll}
% \multicolumn{2}{l}{Initialize \texttt{incMovingCount}}\\
% \texttt{incMovingCount}$(0, T, T{+}L)@T$ & $\leftarrow \mbox{\tt Src}(V,0)@T$\\
% \texttt{incMovingCount}$(1, T, T{+}L)@T$ & $\leftarrow \mbox{\tt Src}(V,1)@T$\\
% & \\
% \multicolumn{2}{l}{Propagate \texttt{incMovingCount}}\\
% \texttt{incMovingCount}$(A, S, E)@(T{+}1)$ & $\leftarrow$ \texttt{incMovingCount}$(A, S, E)@T ,\  \mbox{\tt Src}(V,0)@(T{+}1),\ (T{+}1{\leq}E)$\\
% \texttt{incMovingCount}$(A{+}1, S, E)@(T{+}1)$ & $\leftarrow$ \texttt{incMovingCount}$(A, S, E)@T ,\  \mbox{\tt Src}(V,1)@(T{+}1),\ (T{+}1{\leq}E)$\\
% & \\
% \multicolumn{2}{l}{Report \texttt{resultMovingCount}}\\
% \texttt{resultMovingCount}$(A, S, E)@T$ & $\leftarrow$ \texttt{incMovingCount}$(A, S, E)@T ,\  (T=E)$\\
% \end{tabular}
% \end{small}
% \end{mdframed}
% \caption{Datalog Program \texttt{incMovingCount} and \texttt{resultMovingCount}}
%     \label{fig:moving-count-infty}
% \end{figure}

% \subsection{Start-triggered window for {\sc max} function}

% Let \texttt{Src} be the source event,
% $L$ the window size,
% and \texttt{Start} a dataless start trigger event
% for a start-triggered rule with the {\sc max} function,
% i.e., the rule has the form of Fig.\:\ref{fig:start-trigger-max-rule}.

% \begin{figure}[h!]
% \begin{tabular}{ll}
% \texttt{resultStartTriggerMax}$(\mbox{\sc max}(a),s)@(s{+}L) \leftarrow$ & $\textsc{Over}({\tt Start}@s, s+L)$\\
%                             & $\textsc{From}\ \mbox{\tt Src}(a)@x$\\
% \end{tabular}
% \caption{Start-triggered rule with {\sc max} function}
% \label{fig:start-trigger-max-rule}
% \end{figure}

% We use a Datalog program (Fig.\:\ref{fig:start-trigger-max-program}) to generate
% \texttt{incStartTriggerMax} events as follows:
% Whenever a start trigger event is observed,
% an \texttt{incStartTriggerMax} event is initialized
% using the start trigger event's timestamp $T$ as the window start,
% the source event's value in the {\em accumulator},
% and $T{+}L$ as the window end.
% Data in \texttt{incStartTriggerMax} at a timestamp $T$ is propagated to the next timestamp $T{+}1$ using the source event at timestamp $T{+}1$.
% To do this, if the source event is real,
% the accumulator is compared with the value of the source event
% at timestamp $T{+}1$ and the larger is passed along.
% If the source event is not real,
% the accumulator is passed along.
% The propagation only happens if the next timestamp is no greater than the window end.
% When the source timestamp is equal to the window end,
% the value in the accumulator is reported with \texttt{reportStartTriggerMax}.\\

% \begin{figure}[h!]
% \begin{mdframed}[leftmargin=0pt,rightmargin=0mm]
% \begin{small}
% \begin{tabular}{ll}
% \multicolumn{2}{l}{Initialize \texttt{incStartTriggerMax}}\\
% \texttt{incStartTriggerMax}$(V, T, T{+}L)@T$ & $\leftarrow \mbox{\tt Src}(V,1)@T,\ \texttt{Start}@(T)$\\
% \texttt{incStartTriggerMax}$(-\infty, T, T{+}L)@T$ & $\leftarrow \mbox{\tt Src}(V,0)@T,\ \texttt{Start}@(T)$\\
% & \\
% \multicolumn{2}{l}{Propagate \texttt{incStartTriggerMax}}\\
% \texttt{incStartTriggerMax}$(V, S, E)@(T{+}1)$ & $\leftarrow$ \texttt{incStartTriggerMax}$(A, S, E)@T ,\  \mbox{\tt Src}(V,1)@(T{+}1) ,\  (A{<} V),\ (T{+}1 {\leq} E)$\\
% \texttt{incStartTriggerMax}$(A, S, E)@(T{+}1)$ & $\leftarrow$ \texttt{incStartTriggerMax}$(A, S, E)@T ,\  \mbox{\tt Src}(V,1)@(T{+}1) ,\  (A{\geq}V),\ (T{+}1 {\leq} E)$\\
% \texttt{incStartTriggerMax}$(A, S, E)@(T{+}1)$ & $\leftarrow$ \texttt{incStartTriggerMax}$(A, S, E)@T ,\  \mbox{\tt Src}(V,0)@(T{+}1) ,\ (T{+}1 {\leq} E)$\\
% & \\
% \multicolumn{2}{l}{Report \texttt{resultStartTriggerMax}}\\
% \texttt{resultStartTriggerMax}$(A, S, E)@T$ & $\leftarrow$ \texttt{incStartTriggerMax}$(A, S, E)@T,\ (T{=}E)$\\
% \end{tabular}
% \end{small}
% \end{mdframed}
% \caption{Datalog Program \texttt{incStartTriggerMax} and \texttt{resultStartTriggerMax}}
%     \label{fig:start-trigger-max-program}
% \end{figure}

% \subsection{End-triggered window for {\sc max} function}\label{end-triggered-algorithm}

% Let \texttt{Src} be the source event,
% $L+1$ the window size,
% and \texttt{End} a dataless end trigger event
% for an end-triggered rule with the {\sc max} function,
% i.e., the rule has the form of Fig.\:\ref{fig:end-trigger-max-rule}.

% \begin{figure}[h!]
% \begin{tabular}{ll}
% \texttt{resultEndMax}$(\mbox{\sc max}(a),e-L,e)@e \leftarrow$ & $\textsc{Over}(e-L,\texttt{End}@e)$\\
%                             & $\textsc{From}\ \mbox{\tt Src}(a)@x$\\
% \end{tabular}
% \caption{Rule $r$ with the {\sc max} function, end-triggered window $[e-L,e]$}
% \label{fig:end-trigger-max-rule}
% \end{figure}

% We use a Datalog program (Fig.\:\ref{fig:end-trigger-max})
% to generate \texttt{incEndMax} events as follows:
% each timestamp is possibly the start of an end-triggered window
% so for each source event at timestamp $T$,
% an \texttt{incEndMax} event is initialized
% using a source event's timestamp as the window start,
% the source event's value in the accumulator,
% and $T{+}L$ as the window end.
% Data in \texttt{incEndMax} at a timestamp $T$ is propagated to the next timestamp $T{+}1$ using the source event at timestamp $T{+}1$.
% To do this, if the source event is real,
% the accumulator is compared with the value of the source event
% at timestamp $T{+}1$ and the larger is passed along.
% If the source event is not real,
% the accumulator is passed along.
% The propagation only happens if the next timestamp is no greater than the window end.
% We report the value of the accumulator with \texttt{resultEndMax}
% when window's end stored in \texttt{incEndMax} coincides
% with an \texttt{End} event.\\

% \begin{figure}[h!]
% \begin{mdframed}[leftmargin=0pt,rightmargin=0mm]
% \begin{small}
% \begin{tabular}{ll}
% \multicolumn{2}{l}{Initialize \texttt{incEndMax}}\\
% \texttt{incEndMax}$(V, T, T{+}L)@T$ & $\leftarrow \mbox{\tt Src}(V,1)@T$\\
% \texttt{incEndMax}$(-\infty, T, T{+}L)@T$ & $\leftarrow \mbox{\tt Src}(V,0)@T$\\
% & \\
% \multicolumn{2}{l}{Propagate \texttt{incEndMax}}\\
% \texttt{incEndMax}$(V, S, E)@(T{+}1)$ & $\leftarrow$ \texttt{incEndMax}$(A, S, E)@T ,\  \mbox{\tt Src}(V)@(T{+}1) ,\  (A{<}   V),\ (T{+}1 {\leq} E)$\\
% \texttt{incEndMax}$(A, S, E)@(T{+}1)$ & $\leftarrow$ \texttt{incEndMax}$(A, S, E)@T ,\  \mbox{\tt Src}(V)@(T{+}1) ,\  (A{\geq}V),\ (T{+}1 {\leq} E)$\\
% & \\
% \multicolumn{2}{l}{Report \texttt{resultEndMax}}\\
% \texttt{resultEndMax}$(A, S, E)@T$ & $\leftarrow$ \texttt{incEndMax}$(A, S, E)@T ,\  (T{=}E),\ \texttt{End}@T$\\
% \end{tabular}
% \end{small}
% \end{mdframed}
% \caption{Datalog program for \texttt{incEndMax} and \texttt{resultEndMax}}
%     \label{fig:end-trigger-max}
% \end{figure}

% \end{section}